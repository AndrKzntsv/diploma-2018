% !TeX spellcheck = ru_RU_yo
% !TEX program = xelatex

\documentclass{../../scs-iam}

\usepackage{calc}

\newlength{\remaining}
\newcommand{\titledline}[1]{%
  \setlength{\remaining}{\textwidth-\widthof{#1}-5pt}
  #1
}

\begin{document}
  \pagestyle{headcenter}
  
  \newgeometry{
  top=20mm,
  right=10mm,
  bottom=20mm,
  left=20mm,
  bindingoffset=0cm
}

\restoregeometry

  \setcounter{page}{1}
  
  \mysection*{ВВЕДЕНИЕ}
  
  В настоящее время высокоточное позиционирование является необходимым инструментом для решения огромного количества задач. Автоматизация сельскохозяйственных работ или топографические съёмки служат отличным пример работ, которые невозможно осуществить, используя стандартные GPS-приёмники, позволяющие определять местоположение лишь с дециметровой точностью. Высокоточные же координаты возможно получить, используя технологию \textit{дифференциального GPS}. Однако, данное решение подразумевает использование сложных алгоритмов, а~стоимость представленных на рынке устройств, позволяющих производить подобные расчёты, может превышать $10000$ долларов США. \par
  
  Для тех, кому по тем или иным причинам дорогостоящее оборудование недоступно, решением может стать RTKLIB -- проект с~открытым исходным кодом, реализующий вышеупомянутые алгоритмы для стандартных, общедоступных приёмников. Однако, распространению данного пакета программ мешает неудобство его использования: для управления и~мониторинга требуется наличие полноценного компьютера, а~программы RTKLIB имеют множество режимов работы и~настроек, что достаточно сильно повышает общий порог вхождения. \par
  
  В рамках данной практики работа осуществлялась с устройствами Emlid Reach и Emlid ReachRS. Основой программного обеспечения данных продуктов является вышеупомянутый программный комплекс. Данные устройства облегчают работу с RTKLIB с помощью специального веб-приложения, которое позволяет настраивать комплекс и проводить геодезические работы, используя любое устройство с веб-браузером.
\end{document}
