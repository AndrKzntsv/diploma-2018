% !TeX spellcheck = ru_RU_yo
% !TEX program = xelatex

\documentclass[ltitles]{../../scs-iam}

\usepackage{calc}

\newlength{\remaining}
\newcommand{\titledline}[1]{%
  \setlength{\remaining}{\textwidth-\widthof{#1}-5pt}
  #1
}

\begin{document}
  \pagestyle{headcenter}
  
  \newgeometry{
  top=20mm,
  right=15mm,
  bottom=20mm,
  left=20mm,
  bindingoffset=0cm
}

\renewcommand{\strut}{\rule[-.1\baselineskip]{0pt}{\baselineskip}}

\thispagestyle{empty}

\begin{center}
  {
    \bfseries
    {
      \subnormal
      Министерство образования и науки Российской Федерации
    } \\[-0.5em]
    {
      \scriptsize
      ФЕДЕРАЛЬНОЕ ГОСУДАРСТВЕННОЕ АВТОНОМНОЕ ОБРАЗОВАТЕЛЬНОЕ УЧРЕЖДЕНИЕ ВЫСШЕГО ОБРАЗОВАНИЯ
    } \\[-0.25em]
    {
      \subnormal
      “САНКТ-ПЕТЕРБУРГСКИЙ НАЦИОНАЛЬНЫЙ ИССЛЕДОВАТЕЛЬСКИЙ \\[-0.5em]
      УНИВЕРСИТЕТ ИНФОРМАЦИОННЫХ ТЕХНОЛОГИЙ, \\[-0.75em]
      МЕХАНИКИ И ОПТИКИ”
    } \\[0.25em]
    {
      \normalsize
      ПОЯСНИТЕЛЬНАЯ ЗАПИСКА \\[-0.5em]
      ВЫПУСКНОЙ КВАЛИФИКАЦИОННОЙ РАБОТЫ
    } \\[5.75em]
    {
      \normalsize
      <<РАЗРАБОТКА ВЕБ-ПРИЛОЖЕНИЯ ДЛЯ РАБОТЫ С ПРОГРАММНЫМ \\[-0.5em]
      ПАКЕТОМ ВЫСОКОТОЧНОГО ПОЗИЦИОНИРОВАНИЯ RTKLIB>>
    } \\[6.75em]
  }
\end{center}

\begin{flushright}
  {
    \small
    \begin{minipage}{.8\textwidth}
      Автор $\underset{\text{\scriptsize (фамилия, имя, отчество)}}{\underline{\makebox[.65\textwidth][s]{\strut\hfill Кузнецов Андрей Андреевич\hfill}}}$
      \hfill
      $\underset{\text{\scriptsize (подпись)}}{\underline{\strut\hspace{.25\textwidth}}}$ \\[-0.5em]

      Направление подготовки (специальность) 
      \hfill 
      $\underline{\makebox[.47\textwidth][s]{\strut\hfill 00.00.00\hfill}}$ \\[-0.5em]

      Квалификация
      \hfill
      $\underset{\text{\scriptsize (бакалавр, инженер, магистр)}}{\underline{\makebox[.81\textwidth][s]{\strut\hfill магистр\hfill}}}$ \\[-0.5em]

      Руководитель
      \hfill
      $\underset{\text{\scriptsize (Фамилия, И., О.,  ученое звание, степень)}}{\underline{\makebox[.83\textwidth][s]{\strut\hfill Соснин Владимир Валерьевич, к.т.н., доцент\hfill}}}$ \\[3em]

      \textbf{К защите допустить} \\[0.25em]
      Зав.кафедрой $\underset{\text{(Фамилия, И., О.,  ученое звание, степень)}}{\underline{\makebox[.81\textwidth][s]{\strut\hfill Муромцев Дмитрий Ильич, к.т.н., доцент\hfill}}}$
    \end{minipage}
  }
\end{flushright}

\vfill

\begin{center}
  {
    \normalsize
    Санкт-Петербург, 2018 г.
  }
\end{center}

\restoregeometry

  \setcounter{page}{1}
  
  \mysection{Общие сведения}
  
  С 1 по 30 ноября обучающийся проходил производственную практику в Университете ИТМО. На практику было дано задание по разработке программного модуля веб-приложения для управление GPS-приёмником, работающим под управлением программного обеспечения, основанного на программном комплексе высокоточного позиционирования RTKLIB. \par
  
  В процессе прохождения практики были изучены следующие электронные источники и литература:
  \begin{dashitemize}
    \item документация программного комплекса RTKLIB;
    \item документация устройств Emlid Reach и Emlid ReachRS;
    \item техническое задание на разработку программного модуля.
  \end{dashitemize}

  \mysection{Ход работы}
  
  \subsection{Этап 1 -- Знакомство с платформой разработки}
  
  В рамках данной практики платформой для разработки являлись устройства компании Emlid: GPS-модуль Reach и GPS-приёмник ReachRS. Данные устройства работают под управлением программного обеспечения, основанного на программном комплексе высокоточного позиционирования RTKLIB. Работа пользователя с данными продуктами осуществляется через веб-приложение, доступ к которому можно получить с помощью любого устройства, на котором установлен современный веб-браузер. \par
  
  Веб-клиент рассматриваемых устройств написан на языках Python (pexpect, Flask) и JavaScript (jQuery, Vue.js, D3.js, OpenLayers).
  
  \subsection{Этап 2 -- Постановка задачи}
  
  Основной задачей производственной практики являлось создание программного компонента, необходимого для проведения геодезических изысканий с помощью вышеупомянутых GPS-приёмников. \par
  
  Также ставится задача встраивания рассматриваемого программного модуля в существующее веб-приложение, через которое осуществляется вся работа с приёмником.
  
  \subsection{Этап 3 -- Разработка модуля}
  
  \subsection{Этап 4 -- Тестирование модуля}
  
  \subsection{Этап 5 -- Оформление пользовательской документации}
\end{document}
