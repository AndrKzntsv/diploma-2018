% !TeX spellcheck = ru_RU_yo
% !TEX program = xelatex

\documentclass{../../scs-iam}

\usepackage{calc}

\newlength{\remaining}
\newcommand{\titledline}[1]{%
  \setlength{\remaining}{\textwidth-\widthof{#1}-5pt}
  #1
}

\begin{document}
  \pagestyle{headcenter}
  
  \newgeometry{
  top=20mm,
  right=15mm,
  bottom=20mm,
  left=20mm,
  bindingoffset=0cm
}

\renewcommand{\strut}{\rule[-.1\baselineskip]{0pt}{\baselineskip}}

\thispagestyle{empty}

\begin{center}
  {
    \bfseries
    {
      \subnormal
      Министерство образования и науки Российской Федерации
    } \\[-0.5em]
    {
      \scriptsize
      ФЕДЕРАЛЬНОЕ ГОСУДАРСТВЕННОЕ АВТОНОМНОЕ ОБРАЗОВАТЕЛЬНОЕ УЧРЕЖДЕНИЕ ВЫСШЕГО ОБРАЗОВАНИЯ
    } \\[-0.25em]
    {
      \subnormal
      “САНКТ-ПЕТЕРБУРГСКИЙ НАЦИОНАЛЬНЫЙ ИССЛЕДОВАТЕЛЬСКИЙ \\[-0.5em]
      УНИВЕРСИТЕТ ИНФОРМАЦИОННЫХ ТЕХНОЛОГИЙ, \\[-0.75em]
      МЕХАНИКИ И ОПТИКИ”
    } \\[0.25em]
    {
      \normalsize
      ПОЯСНИТЕЛЬНАЯ ЗАПИСКА \\[-0.5em]
      ВЫПУСКНОЙ КВАЛИФИКАЦИОННОЙ РАБОТЫ
    } \\[5.75em]
    {
      \normalsize
      <<РАЗРАБОТКА ВЕБ-ПРИЛОЖЕНИЯ ДЛЯ РАБОТЫ С ПРОГРАММНЫМ \\[-0.5em]
      ПАКЕТОМ ВЫСОКОТОЧНОГО ПОЗИЦИОНИРОВАНИЯ RTKLIB>>
    } \\[6.75em]
  }
\end{center}

\begin{flushright}
  {
    \small
    \begin{minipage}{.8\textwidth}
      Автор $\underset{\text{\scriptsize (фамилия, имя, отчество)}}{\underline{\makebox[.65\textwidth][s]{\strut\hfill Кузнецов Андрей Андреевич\hfill}}}$
      \hfill
      $\underset{\text{\scriptsize (подпись)}}{\underline{\strut\hspace{.25\textwidth}}}$ \\[-0.5em]

      Направление подготовки (специальность) 
      \hfill 
      $\underline{\makebox[.47\textwidth][s]{\strut\hfill 00.00.00\hfill}}$ \\[-0.5em]

      Квалификация
      \hfill
      $\underset{\text{\scriptsize (бакалавр, инженер, магистр)}}{\underline{\makebox[.81\textwidth][s]{\strut\hfill магистр\hfill}}}$ \\[-0.5em]

      Руководитель
      \hfill
      $\underset{\text{\scriptsize (Фамилия, И., О.,  ученое звание, степень)}}{\underline{\makebox[.83\textwidth][s]{\strut\hfill Соснин Владимир Валерьевич, к.т.н., доцент\hfill}}}$ \\[3em]

      \textbf{К защите допустить} \\[0.25em]
      Зав.кафедрой $\underset{\text{(Фамилия, И., О.,  ученое звание, степень)}}{\underline{\makebox[.81\textwidth][s]{\strut\hfill Муромцев Дмитрий Ильич, к.т.н., доцент\hfill}}}$
    \end{minipage}
  }
\end{flushright}

\vfill

\begin{center}
  {
    \normalsize
    Санкт-Петербург, 2018 г.
  }
\end{center}

\restoregeometry

  \setcounter{page}{1}
  
  \mysection*{ВВЕДЕНИЕ}
  
  В настоящее время высокоточное позиционирование является необходимым инструментом для решения огромного количества задач. Автоматизация сельскохозяйственных работ или топографические съёмки служат отличным пример работ, которые невозможно осуществить, используя стандартные GPS-приёмники, позволяющие определять местоположение лишь с дециметровой точностью. Высокоточные же координаты возможно получить, используя технологию \textit{дифференциального GPS}. Однако, данное решение подразумевает использование сложных алгоритмов, а~стоимость представленных на рынке устройств, позволяющих производить подобные расчёты, может превышать $10000$ долларов США. \par
  
  Для тех, кому по тем или иным причинам дорогостоящее оборудование недоступно, решением может стать RTKLIB -- проект с~открытым исходным кодом, реализующий вышеупомянутые алгоритмы для стандартных, общедоступных приёмников. Однако, распространению данного пакета программ мешает неудобство его использования: для управления и~мониторинга требуется наличие полноценного компьютера, а~программы RTKLIB имеют множество режимов работы и~настроек, что достаточно сильно повышает общий порог вхождения. \par
  
  В рамках данной практики работа осуществлялась с устройствами Emlid Reach и Emlid ReachRS. Основой программного обеспечения данных продуктов является вышеупомянутый программный комплекс. Данные устройства облегчают работу с RTKLIB с помощью специального веб-приложения, которое позволяет настраивать комплекс и проводить геодезические работы, используя любое устройство с веб-браузером.
\end{document}
