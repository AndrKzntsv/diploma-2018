% !TeX spellcheck = ru_RU_yo
% !TEX program = xelatex

\documentclass[ltitles]{../../scs-iam}

\usepackage{calc}

\newlength{\remaining}
\newcommand{\titledline}[1]{%
  \setlength{\remaining}{\textwidth-\widthof{#1}-5pt}
  #1
}

\begin{document}
  \pagestyle{headcenter}
  
  \newgeometry{
  top=20mm,
  right=10mm,
  bottom=20mm,
  left=20mm,
  bindingoffset=0cm
}

\restoregeometry

  \setcounter{page}{1}
  
  \mysection{Общие сведения}
  
  С 1 по 30 ноября обучающийся проходил производственную практику в Университете ИТМО. На практику было дано задание по разработке программного модуля веб-приложения для управление GPS-приёмником, работающим под управлением программного обеспечения, основанного на программном комплексе высокоточного позиционирования RTKLIB. \par
  
  В процессе прохождения практики были изучены следующие электронные источники и литература:
  \begin{dashitemize}
    \item документация программного комплекса RTKLIB;
    \item документация устройств Emlid Reach и Emlid ReachRS;
    \item техническое задание на разработку программного модуля.
  \end{dashitemize}

  \mysection{Ход работы}
  
  \subsection{Этап 1 -- Знакомство с платформой разработки}
  
  В рамках данной практики платформой для разработки являлись устройства компании Emlid: GPS-модуль Reach и GPS-приёмник ReachRS. Данные устройства работают под управлением программного обеспечения, основанного на программном комплексе высокоточного позиционирования RTKLIB. Работа пользователя с данными продуктами осуществляется через веб-приложение, доступ к которому можно получить с помощью любого устройства, на котором установлен современный веб-браузер. \par
  
  Веб-клиент рассматриваемых устройств написан на языках Python (pexpect, Flask) и JavaScript (jQuery, Vue.js, D3.js, OpenLayers).
  
  \subsection{Этап 2 -- Постановка задачи}
  
  Основной задачей производственной практики являлось создание программного компонента, необходимого для проведения геодезических изысканий с помощью вышеупомянутых GPS-приёмников. \par
  
  Также ставится задача встраивания рассматриваемого программного модуля в существующее веб-приложение, через которое осуществляется вся работа с приёмником.
  
  \subsection{Этап 3 -- Разработка модуля}
  
  \subsection{Этап 4 -- Тестирование модуля}
  
  \subsection{Этап 5 -- Оформление пользовательской документации}
\end{document}
