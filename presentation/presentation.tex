% !TEX program = xelatex
% !TeX spellcheck = ru_RU_yo

\documentclass[xetex,с,aspectratio=169]{beamer}

\usepackage{xecyr}
\usepackage{xunicode}
\usepackage{fontspec}
\defaultfontfeatures{Ligatures=TeX}
\setmainfont{ALSSchlangesans}

\usepackage{polyglossia}
\setdefaultlanguage[spelling=modern]{russian}
\newfontfamily{\cyrillicfont}[BoldFont={*-Bold}]{ALSSchlangesans}
\newfontfamily{\cyrillicfontsf}[BoldFont={*-Bold}]{ALSSchlangesans}
\newfontfamily{\cyrillicfonttt}{Droid Sans Mono}

\usepackage[euler-digits,small]{eulervm}
\AtBeginDocument{\renewcommand{\hbar}{\hslash}}

\definecolor{ifmoblue}{RGB}{25,70,186}
\definecolor{ifmored}{RGB}{236,11,67}

\setbeamertemplate{itemize items}[circle]

\usepackage{epstopdf}
\usepackage{graphicx}
\usepackage{amsmath}
\usepackage{nicefrac}
\usepackage{ctable}
\usepackage{ragged2e}

\usepackage{pifont}
\newcommand{\cmark}{\ding{51}}%
\newcommand{\xmark}{\ding{55}}%


\usepackage{xcolor}
\newcommand{\highlight}[1]{\colorbox{orange!50}{$\displaystyle#1$}}

\usepackage{textpos}

\newcommand{\email}[1]{{\scriptsize\texttt{#1}}}

\usebackgroundtemplate{}

\titlegraphic{\includegraphics[width=5cm]{ifmo/logo-blue}}
\title{Разработка веб-приложения для работы с программным пакетом высокоточного позиционирования RTKLIB}
\author[Кузнецов А.А., P3410]{Кузнецов Андрей Андреевич, ФПИиКТ, ИПМ, Р4215}
\date[]{Научный руководитель: Соснин В.В., к.т.н., доцент}

\usetheme{ifmo}
\setbeamersize{text margin left=0.6cm,text margin right=0.5cm}

\begin{document}

%
% Title
%
{
  \setbeamertemplate{footline}{}
  \begin{frame}
    \titlepage
  \end{frame}
}


%
% DGPS
%
\begin{frame}
  \frametitle{Дифференциальная GPS}

  \begin{center}
    \textbf{Дифференциальная GPS} -- система, предназначенная для повышения\\точности сигналов GPS.
  \end{center}

  \vskip -0.25cm

  \begin{figure}[h]
    \centering
    \includegraphics[height=5.5cm]{../img/tikz/dgps-one/pic}
  \end{figure}
\end{frame}


%
% RTK
%
\begin{frame}
  \frametitle{Кинематика реального времени}

  \begin{center}
    \textbf{Кинематика реального времени} (англ. \emph{Real Time Kinematic, RTK}) -- режим работы, при котором приём и~применение поправок с~базы происходят в~реальном времени.
  \end{center}

  \vskip0.2cm
  \begin{center}
    \vskip -0.7cm
    \color{ifmoblue}{\rule{.5\textwidth}{0.5pt}}
  \end{center}

  \begin{minipage}{\textwidth}
    \centering
    \begin{minipage}[t]{.3\textwidth}
      \centering
      \begin{figure}[h]
        \centering
        \includegraphics[height=21pt]{../img/trimble}
      \end{figure}
      \$~10$\,$000

      {
        \scriptsize
        \color{gray}{Trimble R8 Model 3 (2009)}
      }
    \end{minipage}
    \hspace{1em}
    \begin{minipage}[t]{.3\textwidth}
      \centering
      \begin{figure}[h]
        \centering
        \includegraphics[height=21pt]{../img/leica}
      \end{figure}
      \$~6$\,$000

      {
        \scriptsize
        \color{gray}{Leica Viva GS08 (2012)}
      }
    \end{minipage}
  \end{minipage}
\end{frame}


%
% RTKLIB
%
\begin{frame}
  \frametitle{RTKLIB (\,1\,)}

  \Large

  \begin{center}
    \textbf{RTKLIB} -- программный пакет с~открытым исходным кодом, предназначенный для осуществления стандартного и~высокоточного позиционирования с~помощью глобальных навигационных спутниковых систем.
  \end{center}
  
  \begin{figure}[h]
    \centering
    \includegraphics[height=2cm]{../img/rtklib}
  \end{figure}
\end{frame}


%
% RTKLIB problems
%
\begin{frame}
  \frametitle{RTKLIB (\,2\,)}
  \framesubtitle{Проблемы использования}

  \begin{figure}[h]
    \centering
    \includegraphics[width=.75\textwidth]{../img/rtklib-hell}
  \end{figure}
\end{frame}


%
% Main task
%
\begin{frame}
  \frametitle{Характеристика проведённой работы}
  \vskip 0.5cm
  \begin{center}
    {
      \Large
      \textbf{Предмет исследования} -- процесс взаимодействия пользователя с~программными компонентами пакета RTKLIB.
      \vskip .75cm
      \textbf{Цель работы} -- создание приложения, позволяющего\\[0.35em]взаимодействовать с~RTKLIB через веб-браузер.
    }
  \end{center}
\end{frame}


%
% Receivers review
%
\begin{frame}
  \frametitle{Обзор существующих решений (\,1\,)}
  \framesubtitle{Интерфейсы для управления приёмниками}

  \begin{textblock*}{3cm}(0.5cm,-2.5cm)
    \includegraphics[width=\textwidth]{../img/trimble-tsc3}
  \end{textblock*}
  \begin{textblock*}{3cm}(2cm,2cm)
    \includegraphics[width=\textwidth]{../img/leica-gs16}
  \end{textblock*}
  \begin{textblock*}{1.5cm}(6.5cm,-1cm)
    \includegraphics[width=\textwidth]{../img/javad-mobile-tools}
  \end{textblock*}
  \begin{textblock*}{3cm}(5cm,-0.5cm)
    \includegraphics[width=\textwidth]{../img/javad-victor}
  \end{textblock*}
  \begin{textblock*}{3.25cm}(9.5cm,-2.5cm)
    \includegraphics[width=\textwidth]{../img/javad-netbrowser}
  \end{textblock*}
  \begin{textblock*}{1.5cm}(12cm,1.25cm)
    \includegraphics[width=\textwidth]{../img/drotek-web}
  \end{textblock*}
\end{frame}


%
% Web apps review
%
\begin{frame}
  \frametitle{Обзор существующих решений (\,2\,)}
  \framesubtitle{Веб-интерфейсы для управления устройствами}
  \vskip 1cm
  \begin{minipage}{\textwidth}
    \centering
    \begin{minipage}[c]{.45\textwidth}
      \centering
      \begin{figure}[h]
        \centering
        \includegraphics[width=\textwidth]{../img/openwrt-menu}
      \end{figure}
      OpenWrt
    \end{minipage}
    \hspace{1em}
    \begin{minipage}[c]{.45\textwidth}
      \centering
      \begin{figure}[h]
        \centering
        \includegraphics[width=\textwidth]{../img/win10-device-info}
      \end{figure}
      Windows 10 IoT Core
    \end{minipage}
  \end{minipage}
\end{frame}


%
% Reach & Reach RS
%
\begin{frame}
  \frametitle{Платформа для разработки}
  
  \begin{minipage}{\textwidth}
    \centering
    \includegraphics[width=.25\textwidth]{../img/emlid-logo}
    \vskip 1.25cm
    \begin{minipage}[c]{.45\textwidth}
      \centering
      \begin{figure}[h]
        \centering
        \includegraphics[height=3.5cm]{../img/reach}
      \end{figure}
      Reach
    \end{minipage}
    \hspace{1em}
    \begin{minipage}[c]{.45\textwidth}
      \centering
      \begin{figure}[h]
        \centering
        \includegraphics[height=3.5cm]{../img/reach-rs}
      \end{figure}
      Reach RS
    \end{minipage}
  \end{minipage}
\end{frame}


%
% Required features
%
\begin{frame}
  \frametitle{Основные требования к веб-приложению}
  
  \large
  
  \begin{itemize}
    \item Одностраничное приложение
    \item Автоматическая подстройка под тип устройства
    \item Адаптивность и кроссбраузерность
  \end{itemize}
  \begin{center}
    \vskip -0.7cm
    \color{ifmoblue}{\rule{.5\textwidth}{0.5pt}}
  \end{center}
  \vskip -0.5cm
  \begin{itemize}
    \item \textbf{Возможность производить геодезические изыскания}
    \item Отображение информации в~соответствии с~текущей ролью в~RTK-системе
    \item Настройка RTK и~приёмника
    \item Настройка входных/выходных потоков данных
    \item Доступ к~логам и~их настройкам
    \item Настройка беспроводных интерфейсов
  \end{itemize}
\end{frame}


%
% System architecture
%
\begin{frame}
  \frametitle{Общая архитектура приложения}
  \vskip -0.5cm
  \begin{figure}[h]
    \centering
    \includegraphics[width=.75\textwidth]{../img/tikz/system-architecture/pic_sans_no-border}
  \end{figure}
\end{frame}


%
% FE architecture
%
\begin{frame}
  \frametitle{Архитектура клиентской части приложения}
  \vskip -0.25cm
  \begin{figure}[h]
    \centering
    \includegraphics[width=.9\textwidth]{../img/tikz/fe-architecture/pic}
  \end{figure}
\end{frame}


%
% UI
%
\begin{frame}
  \frametitle{Разработка веб-приложения (\,1\,)}
  \framesubtitle{Адаптивный интерфейс}

  \begin{minipage}{\textwidth}
    \centering
    \begin{minipage}[c]{.5\textwidth}
      \centering
      \begin{figure}[c]
        \centering
        \includegraphics[height=4cm]{../img/reachview/homepage_responsive-lg}
      \end{figure}
    \end{minipage}
    \hspace{2em}
    \begin{minipage}[c]{.3\textwidth}
      \centering
      \begin{figure}[c]
        \centering
        \includegraphics[height=5.5cm]{../img/reachview/homepage_responsive-xs}
      \end{figure}
    \end{minipage}
  \end{minipage}
\end{frame}


\setbeamercovered{transparent}

%
% Tabs
%
\begin{frame}
  \frametitle{Разработка веб-приложения (\,2\,)}
  \framesubtitle{Разделение интерфейса на секции}

  \begin{columns}[T]
%    \centering
%    \footnotesize
%    \vskip -1cm
    \begin{column}[T]{.32\textwidth}
      \begin{itemize}
        \item<1> Статус
        \item<2> Изыскания
        \item<3> Настройки RTK
        \item<4> Входящие поправки
        \item<5> Выдача позиции
        \item<6> Режим базы
        \item<7> Логирование
        \item<8> Управление камерой
        \item<9> Wi-Fi/Bluetooth
        \item<10> Настройки
      \end{itemize}
    \end{column}
    \hspace{1em}
    \begin{column}{.64\textwidth}
      \only<1>{
        \begin{figure}[c]
          \centering
          \includegraphics[height=5.5cm]{../img/reachview/status_content_laptop}
        \end{figure}
      }
      \only<2>{
        \begin{figure}[c]
          \centering
          \includegraphics[height=5.5cm]{../img/reachview/survey_content_laptop}
        \end{figure}
      }
      \only<3>{
        \begin{figure}[c]
          \centering
          \includegraphics[height=4.25cm]{../img/reachview/rtk-settings_content_laptop}
        \end{figure}
      }
      \only<4>{
        \begin{figure}[c]
          \centering
          \includegraphics[height=5.5cm]{../img/reachview/correction-input_content_laptop}
        \end{figure}
      }
      \only<5>{
        \vskip 1cm
        \begin{figure}[c]
          \centering
          \includegraphics[height=3cm]{../img/reachview/position-output_content_laptop}
        \end{figure}
      }
      \only<6>{
        \begin{figure}[c]
          \centering
          \includegraphics[height=5cm]{../img/reachview/base-mode_content_laptop}
        \end{figure}
      }
      \only<7>{
        \begin{figure}[c]
          \centering
          \includegraphics[height=5.5cm]{../img/reachview/logging_content_laptop}
        \end{figure}
      }
      \only<8>{
        \vskip 0.5cm
        \begin{figure}[c]
          \centering
          \includegraphics[height=3.5cm]{../img/reachview/camera-control_content_laptop}
        \end{figure}
      }
      \only<9>{
        \begin{figure}[c]
          \centering
          \includegraphics[height=5.5cm]{../img/reachview/wifi-bt_content_laptop}
        \end{figure}
      }
      \only<10>{
        \begin{figure}[c]
          \centering
          \includegraphics[height=4.75cm]{../img/reachview/settings_content_laptop}
        \end{figure}
      }
    \end{column}
  \end{columns}
\end{frame}


%
% Testing
%
\begin{frame}
  \frametitle{Тестирование приложения}
  {
    \Large
    \begin{itemize}
      \setlength\itemsep{0.75em}
      \item Модульные тесты
      \item Интеграционные тесты
      \item UI-тесты\\[1.25em]
      \item \textbf{Beta-версии приложения для пользователей}\\[0.25em](с отзывами на форуме)
    \end{itemize}
  }
\end{frame}


%
% Results
%
\begin{frame}
  \frametitle{Результаты}
  {
    \large
    \begin{itemize}
      \setlength\itemsep{1em}
      \item[1.] Изучен процесс работы с~GPS-приёмниками в~режиме RTK
      \item[2.] Создано веб-приложение для работы с~программным комплексом RTKLIB, которое соответствует всем предъявленным требованиям
      \item[3.] Созданное приложение протестировано и~внедрено
      \item[4.] Налажен процесс общения с~пользователями, что позволяет получать отзывы и~отчёты об ошибках
      \item[5.] Создано два канала получения обновлений приложения
    \end{itemize}
  }
\end{frame}


%
% The End
%
\begin{frame}[c]
\begin{center}
  \Huge\bfseries
  \color{ifmoblue}{Спасибо за внимание}
\end{center}
\end{frame}

\end{document}
