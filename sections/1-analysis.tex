\mysection{АНАЛИЗ ПРЕДМЕТНОЙ ОБЛАСТИ}

\subsection{Дифференциальная GPS и кинематика реального времени}

Дифференциальная GPS (англ. \emph{Differential Global Positioning System, DGPS}) -- система, предназначенная для повышения точности сигналов GPS. Принцип работы данной системы заключается в~измерении и~учёте при работе разницы между рассчитанной и~закодированной псевдодальностями до спутников. \par

DGPS подразумевает использование двух приёмников при проведении измерений:

\begin{dashitemize}
  \item \textbf{База} -- стационарный приёмник, который находится в точке с заранее рассчитанной координатой. База транслирует данные о разнице между информацией о позиции, полученной со спутника, и закодированными данными о своём местонахождении.
  \item \textbf{Ровер}
\end{dashitemize}

\subsection{Программный пакет RTKLIB}
\subsubsection{Поддерживаемые спутниковые системы}
\subsubsection{Режимы работы}
\subsubsection{Поддерживаемые форматы данных}
\subsubsection{Программы, входящие в состав RTKLIB}

\subsection{Основные проблемы использования RTKLIB}

\subsection{Обзор существующих веб-приложений, предназначенных для работы с устройствами без органов управления}

\subsection{Выводы по разделу 1}

\newpage
