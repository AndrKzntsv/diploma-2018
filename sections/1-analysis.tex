\mysection{АНАЛИЗ ПРЕДМЕТНОЙ ОБЛАСТИ}

\subsection{Особенности работы GPS-приёмников}

Каждый GPS-приёмник определяет свои координаты, основываясь на расстояниях до спутников, с~которых он получает сигналы. Данные расстояния вычисляются из времени, которое требуется радиосигналам для прохождения от космических аппаратов до приёмника. \par

Для установления позиции приёмнику необходимо получать сигналы минимум от четырёх спутников. Каждый из этих сигналов может быть искажён при прохождении через слои атмосферы или при отражении от различных наземных объектов -- данные явления вызывают появление ошибок и~задержек, что отрицательно сказывается на точности позиционирования. \par

Важную роль в~решении проблемы, описанной выше, играет масштабность системы GPS. Расстояние между наземными объектами и~космическими спутниками так велико, что многие расстояния на земле становятся незначительными. Иными словами, если разместить два приёмника на расстоянии нескольких сотен километров друг от друга, то сигналы, которые они будут получать со спутников, будут проходить практически через одну и~ту же часть атмосферы, что позволит считать ошибки на обоих приёмниках одинаковыми.

\subsection{Дифференциальная GPS и кинематика реального времени}

Дифференциальная GPS (англ. \emph{Differential Global Positioning System, DGPS}) -- система, предназначенная для повышения точности сигналов GPS. Принцип работы данной системы заключается в~измерении и~учёте при работе разницы между рассчитанной и~закодированной псевдодальностями до спутников. \par

Важнейшей особенность DGPS является использование двух приёмников при проведении измерений:

\begin{dashitemize}
  \item \textbf{База} (англ. \emph{base}) -- стационарный приёмник, который находится в~точке с~заранее рассчитанной координатой. База транслирует данные о~разнице между информацией о~позиции, полученной со спутника, и~закодированными данными о~своём местонахождении.
  \item \textbf{Ровер} (англ. \emph{rover}) -- приёмник, с~помощью которого производятся какие-либо измерения. Используя данные, полученные с~базы, ровер учитывает влияние внешних факторов на расчёт координаты, тем самым получая более точную информацию о~своём местонахождении.
\end{dashitemize}

Таким образом, работа дифференциальной GPS основана на принципе, описанном в~пункте 1.1 -- считая искажения спутниковых сигналов одинаковыми для близлежащих приёмников, мы получаем возможность вносить поправки в~получаемые данные, тем самым улучшая результаты измерений.

\subsection{Программный пакет RTKLIB}
\subsubsection{Поддерживаемые спутниковые системы}
\subsubsection{Режимы работы}
\subsubsection{Поддерживаемые форматы данных}
\subsubsection{Программы, входящие в состав RTKLIB}

\subsection{Основные проблемы использования RTKLIB}

\subsection{Обзор существующих веб-приложений, предназначенных для работы с устройствами без органов управления}

\subsection{Выводы по разделу 1}

\newpage
