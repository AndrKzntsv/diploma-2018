\mysection{АПРОБАЦИЯ РЕЗУЛЬТАТОВ РАЗРАБОТКИ}

\subsection{Установка приложения на модули и приёмники}

Разработанный пользовательский веб-интерфейс в составе Python-при-ложения был установлен на несколько устройств Reach и Reach~RS. Автором работы совместно с разработчиками серверной части приложения было произведено ручное тестирование всех представлений и форм, доступных через веб-приложение.

Работа приложения была протестирована на различных устройствах, работающих по управлением ОС семейства Windows, macOS, Linux, а также на нескольких версиях мобильных систем Android и iOS.

Была проверена доступность приложения при подключении приёмников к Wi-Fi-сетям в качестве клиента, а также при их работе в режиме точки доступа.



\subsection{Полевые испытания устройств}

Разработанный веб-интерфейс был также протестирован при работе устройств под открытым небом. Были проверено поведение приложения при:
\begin{dashitemize}
  \item отключении и подключении антенны во время работы устройства (для модулей Reach);
  \item потере и восстановлении соединения с базой;
  \item намеренном добавлении помех к сигналам, получаемым устройством со спутников.
\end{dashitemize}

При тестировании устройств под открытым небом особое внимание было уделено проверке раздела <<Изыскания>>. Был произведён ручной сбор точек на местности, а также осуществлена проверка работы приложения при использовании правил автоматического сбора точек.



\subsection{Бета-версии приложения}

Устройства Reach были анонсированы в 2015 году, а их разработка была финансирована краудфандинговой кампанией [?]. Первые модули поставлялись пользователям с предустановленным веб-интерфейсом, созданным с помощью библиотек jQuery [?] и jQuery Mobile [?]. Интерфейс имел ограниченную функциональность и позволял:
\begin{dashitemize}
  \item просматривать информацию о местоположении приёмника;
  \item записывать и скачивать логи данных;
  \item управлять подключениями к сетям Wi-Fi;
  \item загружать на устройство конфигурационные файлы RTKLIB.
\end{dashitemize}

Данный веб-интерфейс был необходим для ознакомления пользователей с возможностями ГНСС-модуля. Необходимость создания веб-приложения рассматриваемого в данной работе является результатом развития и роста популярности устройств Reach.

% https://community.emlid.com/t/reachview-2-beta-release/4304
% https://community.emlid.com/t/reachview-beta-v2-2-5/4988
% https://community.emlid.com/t/reachview-v2-2-7/5523
Начиная с ноября 2016 года было начато распространение бета-версии разработанного веб-приложения. Установка данного приложения подразумевала перепрошивку устройства.

Благодаря инициативной группе опытных геодезистов, являющихся пользователями устройств Reach, были собраны отчёты о найденных ошибках и пожелания по доработке приложения.

В связи с выпуском Reach~RS, с февраля 2017 также были получены отзывы об использовании приложения на данных приёмниках.

В марте 2017 года приложение получило статус <<стабильно>> продукта и стало основной рабочей версией веб-интерфейса для Reach и Reach~RS.



\subsection{Выводы по разделу 4}

\begin{dashitemize}
  \item По окончании процесса разработки все функции веб-приложения были тщательно протестированы автором работы и сотрудниками компании Emlid.
  \item Приложение было протестировано в веб-браузерах множества устройств, работающих под управлением различных операционных систем.
  \item Были проведены испытания приложения при работе с устройствами под открытым небом.
  \item Бета-версии приложения были протестированы инициативной группой пользователей, имеющих опыт работы с профессиональным геодезическим оборудованием.
  \item Разработанное приложение было успешно внедрено в качестве основной рабочей версии веб-интерфейса устройств Reach и Reach~RS.
\end{dashitemize}

\newpage