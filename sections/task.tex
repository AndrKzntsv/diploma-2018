\newgeometry{
  top=20mm,
  right=15mm,
  bottom=20mm,
  left=20mm,
  bindingoffset=0cm
}

\thispagestyle{empty}

\begin{center}
  {
    \bfseries
    {
      \subnormal
      Министерство образования и науки Российской Федерации
    } \\[-0.5em]
    {
      \scriptsize
      ФЕДЕРАЛЬНОЕ ГОСУДАРСТВЕННОЕ АВТОНОМНОЕ ОБРАЗОВАТЕЛЬНОЕ УЧРЕЖДЕНИЕ ВЫСШЕГО ОБРАЗОВАНИЯ
    } \\[-0.25em]
    {
      \subnormal
      “САНКТ-ПЕТЕРБУРГСКИЙ НАЦИОНАЛЬНЫЙ ИССЛЕДОВАТЕЛЬСКИЙ \\[-0.5em]
      УНИВЕРСИТЕТ ИНФОРМАЦИОННЫХ ТЕХНОЛОГИЙ, \\[-0.75em]
      МЕХАНИКИ И ОПТИКИ” \\[2em]
    }
  }
\end{center}

\small

\begin{flushright}
  \begin{minipage}{.5\textwidth}
    {
      \hfill\textbf{УТВЕРЖДАЮ}\hfill
    }

    \titledline{Зав.кафедрой}
    $\underline{\makebox[\remaining][s]{\strut\hfill}}$

    \setlength{\remaining}{\textwidth}\addsignatureskip
    $\underset{\text{\scriptsize (ФИО)}}{\underline{\makebox[\remaining][s]{\strut\hfill}}}$
    \hfill\signature \\[-0.5em]

    \hfill\datetemplate \\[0.35em]
  \end{minipage}
\end{flushright}

\begin{center}
  {
    \bfseries
    {
      \normalsize
      З А Д А Н И Е \\
    }
    НА  ВЫПУСКНУЮ  КВАЛИФИКАЦИОННУЮ  РАБОТУ \\[1.5em]
  }
\end{center}

{
  \parindent0pt

  \textbf{Студенту}
  $\underline{\text{\strut Кузнецову А.А.~~}}$
  \hfill
  \textbf{Группа}
  $\underline{\text{\strut P4215~~}}$
  \hfill
  \textbf{Кафедра}
  $\underline{\text{\strut ИПМ~~}}$
  \hfill
  \textbf{Факультет}
  $\underline{\text{\strut ПИиКТ~~}}$ \\[-0.5em]

  \titledline{\textbf{Руководитель}}
  $\underset{
    \text{\scriptsize (Фамилия, Имя, Отчество, ученое звание, степень)}
  }{
    \underline{\makebox[\remaining][s]{\strut Соснин Владимир Валеревич, к.т.н., доцент\hfill}}
  }$ \\[-0.5em]

  \textbf{1 Наименование темы}
  \uline{Разработка веб-приложения для работы с программным пакетом высоко\-точного позиционирования RTKLIB\hfill} \\[-1em]
  
  \titledline{\textbf{Направление подготовки (специальность)}}
  $\underline{
    \makebox[\remaining][s]{\strut 09.04.01 -- Информатика и вычислительная техника\hfill}
  }$ \\[-1em]

  \titledline{\textbf{Направленность (профиль)}}
  $\underline{
    \makebox[\remaining][s]{\strut 09.04.01 -- Математические модели и компьютерное моделирование\hfill}
  }$ \\[-1em]

  \titledline{\textbf{Квалификация}}
  $\underline{
    \makebox[\remaining][s]{\strut магистр\hfill}
  }$ \\[-1em]

  \textbf{2 Срок сдачи студентом законченной работы}\hfill\datetemplate \\[-1em]

  \textbf{3 Техническое задание и исходные данные к работе} \\
  \uline{
    3.1 Изучить возможности программного комплекса RTKLIB;\hfill
  }\\
  \uline{
    3.2 Провести обзор существующих решений, использующих веб-приложения для взаимодействия с устройствами без органов управления;\hfill
  }\\
  \uline{
    3.3 Изучить характеристики платформы для разработки -- ГНСС-модуля Reach и ГНСС-приёмника Reach RS компании Emlid, работающих под управлением программного обеспечения, основанного на RTKLIB;\hfill
  }\\
  \uline{
    3.4 Разработать веб-приложения для взаимодействия с устройствами Reach и Reach RS;\hfill
  }\\
  \uline{
    3.5 Произвести тестирования и апробацию разработанного продукта.\hfill
  }\\[-1em]
}

\restoregeometry

\clearpage

\newgeometry{
  top=20mm,
  right=20mm,
  bottom=20mm,
  left=15mm,
  bindingoffset=0cm
}

\thispagestyle{empty}

{
  \parindent0pt

  \textbf{4 Содержание выпускной квалификационной работы (перечень подлежащих разработке вопросов)} \\
  \uline{
    4.1 Анализ предметной области и обзор существующих решений, использующих веб-приложения для взаимодействия с устройствами без органов управления;\hfill
  }\\
  \uline{
    4.2 Обзор платформы для разработки и проектирование приложения;\hfill
  }\\
  \uline{
    4.3 Описание процесса разработки и тестирования кода приложения;\hfill
  }\\
  \uline{
    4.4 Апробация результатов разработки.\hfill
  }\\[-1em]

  \textbf{5 Перечень графического материала (с указанием обязательного материала)} \\
  \uline{
    Презентация по проделанной работе (в формате PDF)\hfill
  }\\[-1em]

  \textbf{6 Исходные материалы и пособия} \\
  \uline{
    6.1 Takasu T. RTKLIB: An Open Source Program Package for GNSS Positioning [Электронный ресурс] // RTKLIB support information. 2015. - Режим доступа: http://www.rtklib.com/\hfill
  }\\
  \uline{
    6.2 Малютина К.И., Шевчук С.О. Сравнение бесплатной программы RTKLib с коммерческим про\-граммным обеспечением для постобработки ГНСС-измерений // Интерэкспо Гео-Сибирь. 2017. №2. - Режим доступа: https://cyberleninka.ru/article/n/sravnenie-besplatnoy-programmy-rtklib-s-kom
  }\\
  \uline{
    mercheskim-programmnym-obespecheniem-dlya-postobrabotki-gnss-izmereniy\hfill
  }\\
  \uline{
    6.3 Emlid Ltd – Reach [Электронный ресурс] // Официальный сайт компании Emlid. Описание ГНСС модуля Reach. 2017. - Режим доступа: https://emlid.com/reach/\hfill
  }\\
  \uline{
    6.4 Emlid Ltd – Reach RS [Электронный ресурс] // Официальный сайт компании Emlid. Описание ГНСС приёмника Reach RS. 2017. - Режим доступа: https://emlid.com/reachrs/\hfill
  }\\[-1em]

  \textbf{7 Дата выдачи задания} \datetemplate\\[-1em]

  Руководитель ВКР~~\signature\\[-1em]

  Задание принял к исполнению~~\signature\hfill\datetemplate\\
}

\normalsize
\restoregeometry

\clearpage
