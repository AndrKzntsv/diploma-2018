\mysection*{ВВЕДЕНИЕ}

\textbf{Актуальность темы}. В~настоящее время сложно представить жизнь без спутниковой навигации -- данная технология стала неотъемлемой частью деятельности огромного числа людей. Спутниковые системы позволяют легко определить улицу или дом, где находится человек, или же просто помочь в ориентировании на незнакомой местности. Но использование систем навигации не ограничивается только лишь бытовым применением -- данная технология широко применяется для решения задач автоматизации сельскохозяйственных работ, картографии, а также в множестве других областей. \par

Точность современного приёмника средней ценовой категории, в~зависимости от условий, при которых осуществлялось определение местоположения, варьируется от трёх до пяти метров. Для повседневного применения, например, ориентации по городу -- это отличный результат. Однако же, для решения более сложных задач, таких, как перечислены выше, необходимы гораздо более точные данные, которые получают, используя технологию \textit{дифференциального GPS}. Данное решение подразумевает использование сложных алгоритмов, а~представленные на рынке устройства, позволяющие производить подобные расчёты, стоят весьма дорого. \par

Для тех, кому по тем или иным причинам дорогостоящее оборудование недоступно, решение может послужить RTKLIB -- проект с~открытым исходным кодом, реализующий вышеупомянутые алгоритмы для стандартных, общедоступных приёмников. Однако, распространению данного пакета программ мешает неудобство его использования: для управления и~мониторинга требуется наличие полноценного компьютера, а~программы RTKLIB имеют множество режимов работы и~настроек, что достаточно сильно повышает общий порог вхождения. \par

Решение указанных выше проблем использования RTKLIB и посвящена предлагаемая работа. \par

\textbf{Объектом исследования} является программный пакет высокоточного позиционирования RTKLIB. \par

\textbf{Предметом исследования} является процесс взаимодействия пользователя с программными компонентами RTKLIB. \par

\textbf{Целью исследования} является создание приложения, позволяющего взаимодействовать с~RTKLIB через веб-браузер. Под взаимодействием понимается возможность наблюдать статус системы, изменять настройки программы, производить сбор данных, а~также работать с~накопленными логами данных глобальных навигационных спутниковых систем (ГНСС). \par

Для достижения цели исследования были сформулированы следующие задачи: