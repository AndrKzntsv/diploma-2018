\mysection*{ВВЕДЕНИЕ}

\textbf{Актуальность темы}. В~настоящее технология глобальной спутниковой навигации стала неотъемлемой частью деятельности огромного числа людей. Спутниковые системы позволяют определить улицу или дом, где находится человек, или же просто помочь в~ориентировании на незнакомой местности. Но использование систем навигации не ограничивается только бытовым применением -- данная технология активно применяется для решения задач автоматизации сельскохозяйственных работ, топографических съёмок, а~также во~множестве других областей.

Точность современных приёмников, установленных в~смартфонах или автомобильных навигаторах, в~зависимости от условий, при которых осуществлялось определение местоположения, варьируется от трёх до пяти метров \cite{GpsAccuracy,GpsGlonassAccuracy}. Подобный результат подходит для повседневного использования, например, для ориентации по городу. Однако же, для решения задач более сложных, чем перечисленные выше, необходимы гораздо более точные данные, которые получают, используя технологию \textit{дифференциальной GPS}. Данное решение предполагает использование в~приёмниках сложных алгоритмов, а~стоимость представленных на рынке устройств, позволяющих производить подобные расчёты, может превышать $10000$ долларов США \cite{GEOOPTIC,JAVAD}.

Для тех, кому по тем или иным причинам дорогостоящее оборудование недоступно, решением может стать RTKLIB \cite{RTKLIB} -- программный пакет с~открытым исходным кодом, реализующий алгоритмы высокоточного позиционирования и поддерживающий работу с общедоступными одночастотными приёмниками. Однако, распространению данного пакета программ мешает неудобство его использования: для мониторинга и~управления необходимо наличие стационарного компьютера или ноутбука, а~программы RTKLIB имеют множество настроек, режимов работы и~конфигурационных файлов, что, соответственно, повышает порог вхождения.

\textbf{Объектом исследования} является программный пакет высокоточного позиционирования RTKLIB.

\textbf{Предметом исследования} является процесс взаимодействия пользователя с~программными компонентами RTKLIB.

\textbf{Целью исследования} является создание веб-приложения для обеспечения взаимодействия пользователя с~программным пакетом RTKLIB, используемом во~встраиваемом решении, с~помощью кроссплатформенного графического интерфейса. Под взаимодействием подразумевается возможность наблюдать за состоянием RTK-системы, изменять настройки компонентов RTKLIB, работать с~накопленными файлами логов, а~также производить геодезические изыскания и~сбор данных.

Для достижения цели исследования был сформулирован следующий ряд \textbf{задач}:

\begin{dashitemize}
  \item изучить состав и~возможности программного комплекса RTKLIB;
  \item произвести анализ существующих веб-приложений, предназначенных для работы устройствами, у~которых отсутствую органы управления;
  \item осуществить проектирование и~разработку приложения;
  \item произвести тестирование приложения.
\end{dashitemize}

\textbf{Средствами разработки} в~представленной работе являются: язык гипертекстовой разметки HTML, язык описания стилей CSS, язык программирования JavaScript, открытые JavaScript-библиотеки D3.js и~OpenLayers, JavaScript-фреймворк Vue.js. Для организации обмена данными серверной и~клиентской частей приложения в~реальном времени используются библиотека Socket.IO, принцип работы которой основан на протоколе WebSocket.

\textbf{Методологической основой} работы послужила гибкая методология разработки (англ. \emph{Agile software development}), ориентированная на итеративный процесс создания программного продукта и~учитывающая возможность динамического формирования требований.

\textbf{Новизна} работы обусловлена отсутствием в~настоящее время каких-либо программных продуктов, позволяющих взаимодействовать через веб-браузер с~геодезическим оборудованием, программное обеспечение которого основано на RTKLIB.

\textbf{Результатом} данной работы является рабочая версия приложения, в~которой реализованы все необходимые функции, перечисленные в~постановке цели исследования. Также была создана и~выложена в~открытый доступ пользовательская документация, поясняющая основные моменты работы с~приложением.

\textbf{Апробация результатов работы}. Наличие документации позволило осуществить открытое тестирование приложения пользователями и, как результат, получить отзывы, сообщения об ошибках и~пожелания к~функциональности.

\newpage
