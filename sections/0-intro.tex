\mysection*{ВВЕДЕНИЕ}

\textbf{Актуальность темы}. В~настоящее время сложно представить жизнь без спутниковой навигации. Данная технология стала неотъемлемой частью повседневной деятельности миллионов людей. GPS, GLONASS и~другие системы позволяют легко определить, например, улицу или дом, где находится человек. Точность современного приёмника средней ценовой категории варьируется от трёх до пяти метров, в~зависимости от условий, при которых осуществлялось определение местоположения. Для бытового применения, например, ориентации по городу -- это отличный результат, при котором человек без труда сможет сориентироваться. \par

Спутниковые системы навигации также имеют множество других применений, таких как автоматизация сельскохозяйственных работ или картография. Для решения подобных задач необходимы гораздо более точные данные, которые получают, используя технологию \textit{дифференциального GPS}. Данное решение подразумевает использование сложных алгоритмов, а~представленные на рынке устройства, позволяющие производить подобные расчёты, стоят весьма дорого. \par

Однако, существует RTKLIB -- проект с~открытым исходным кодом, реализующий вышеупомянутые алгоритмы для стандартных, общедоступных приёмников. Распространению пакета программ RTKLIB мешает неудобство его использования: для управления и~мониторинга требуется наличие полноценного компьютера, а~программы пакета имеют множество режимов работы и~настроек, что достаточно сильно повышает общий порог вхождения. \par

\textbf{Объектом исследования} является программный пакет высокоточного позиционирования RTKLIB. \par

\textbf{Предметом исследования} являются методы взаимодействия пользователя с программными компонентами RTKLIB. \par

\textbf{Целью исследования} является создание приложения, позволяющего взаимодействовать с RTKLIB через веб-браузер. Под взаимодействием понимается возможность наблюдать статус системы, изменять настройки программы, а также работать с накопленными логами данных глобальных навигационных спутниковых систем (ГНСС). \par