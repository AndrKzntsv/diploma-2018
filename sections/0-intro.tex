\mysection*{ВВЕДЕНИЕ}

\textbf{Актуальность темы}. В~настоящее время сложно представить жизнь без спутниковой навигации -- данная технология стала неотъемлемой частью деятельности огромного числа людей. Спутниковые системы позволяют легко определить улицу или дом, где находится человек, или же просто помочь в ориентировании на незнакомой местности. Но использование систем навигации не ограничивается только лишь бытовым применением -- данная технология активно применяется для решения задач автоматизации сельскохозяйственных работ, топографических съёмок, а~также в~множестве других областей. \par

Точность современных приёмников, установленных, например, в~смартфонах или автомобильных навигаторах, в~зависимости от условий, при которых осуществлялось определение местоположения, варьируется от трёх до пяти метров. Для повседневного применения, например, ориентации по городу -- это отличный результат. Однако же, для решения задач более сложных, чем перечисленные выше, необходимы гораздо более точные данные, которые получают, используя технологию \textit{дифференциального GPS}. Данное решение подразумевает использование сложных алгоритмов, а~стоимость представленных на рынке устройств, позволяющих производить подобные расчёты, может превышать $10000$ долларов США. \par

Для тех, кому по тем или иным причинам дорогостоящее оборудование недоступно, решением может стать RTKLIB -- проект с~открытым исходным кодом, реализующий вышеупомянутые алгоритмы для стандартных, общедоступных приёмников. Однако, распространению данного пакета программ мешает неудобство его использования: для управления и~мониторинга требуется наличие полноценного компьютера, а~программы RTKLIB имеют множество режимов работы и~настроек, что достаточно сильно повышает общий порог вхождения. \par

\textbf{Объектом исследования} является программный пакет высокоточного позиционирования RTKLIB. \par

\textbf{Предметом исследования} является процесс взаимодействия пользователя с~программными компонентами RTKLIB. \par

\textbf{Целью исследования} является создание приложения, позволяющего взаимодействовать с~RTKLIB через веб-браузер. Под взаимодействием понимается возможность наблюдать различные статусы и изменять настройки компонентов RTKLIB, производить сбор данных, а~также работать с~накопленными файлами логов данных глобальных навигационных спутниковых систем (ГНСС). \par

Для достижения цели исследования был сформулирован следующий ряд \textbf{задач}:

\begin{dashitemize}
  \item изучить состав и~возможности программного комплекса RTKLIB;
  \item произвести анализ существующих веб-приложений, предназначенных для работы устройствами, у~которых отсутствую органы управления;
  \item осуществить проектирование и разработку приложения;
  \item произвести тестирование приложения.
\end{dashitemize}

Также, по завершении разработки, ставится задача создания открытого программного интерфейса приложения (англ. \emph{Application Programming Interface, API}), с помощью которого пользователи смогут без труда расширять функциональность приложения в соответствии со своими задачами. \par

\textbf{Средствами разработки} в представленной работе являются: языки программирования Python и JavaScript для реализации серверной (англ. \emph{back-end}) и клиентской (англ. \emph{front-end}) частей приложения соответственно, открытые JavaScript-библиотеки D3.js, OpenLayers, JavaScript-фреймворк Vue.js. Для организации обмена данными серверной и клиентской частей приложения в реальном времени используются библиотека Socket.IO, принцип работы которой основывается на протоколе WebSocket. \par

\textbf{Методологической основой} работы послужила гибкая методология разработки (англ. \emph{Agile software development}), ориентированная на итеративный процесс создания программного продукта и учитывающая возможность динамического формирования требований. \par

\textbf{Новизна} работы обусловлена отсутствием в настоящее время программных продуктов с открытым API, основанных на RTKLIB и позволяющих работать с геодезическим оборудованием через веб-браузер. \par

\textbf{Результатом} данной работы является рабочая версия приложения, в котором реализованы все необходимые функции, перечисленные в постановке цели исследования, а также пользовательская документация, выложенная в открытый доступ. Открытый API находится в стадии реализации. \par

\textbf{Апробация результатов работы}. Наличие документации позволило осуществить открытое тестирование приложения пользователями и, как результат, получить отзывы, сообщения об ошибках и пожелания к функциональности. \par

% Автор планирует продолжать работу над проектом и развивать его, добавляя новые функции и исправляя возможные ошибки, которые могут быть найдены во время эксплуатации приложения.
