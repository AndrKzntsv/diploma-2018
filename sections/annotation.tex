\newgeometry{
  top=20mm,
  right=15mm,
  bottom=20mm,
  left=20mm,
  bindingoffset=0cm
}

\thispagestyle{empty}

\begin{center}
  {
    \bfseries
    {
      \subnormal
      Министерство образования и науки Российской Федерации
    } \\[-0.5em]
    {
      \scriptsize
      ФЕДЕРАЛЬНОЕ ГОСУДАРСТВЕННОЕ АВТОНОМНОЕ ОБРАЗОВАТЕЛЬНОЕ УЧРЕЖДЕНИЕ ВЫСШЕГО ОБРАЗОВАНИЯ
    } \\[-0.25em]
    {
      \subnormal
      “САНКТ-ПЕТЕРБУРГСКИЙ НАЦИОНАЛЬНЫЙ ИССЛЕДОВАТЕЛЬСКИЙ \\[-0.5em]
      УНИВЕРСИТЕТ ИНФОРМАЦИОННЫХ ТЕХНОЛОГИЙ, \\[-0.75em]
      МЕХАНИКИ И ОПТИКИ” \\[1em]
    }
  }
\end{center}

\small

\begin{center}
  {
    \bfseries
    {
      \large
      АННОТАЦИЯ \\
    }
    НА  ВЫПУСКНУЮ  КВАЛИФИКАЦИОННУЮ  РАБОТУ \\[1.5em]
  }
\end{center}

{
  \parindent0pt

  \titledline{\textbf{Студент}}
  $\underset{
    \text{\scriptsize (Фамилия, Имя, Отчество)}
  }{
    \underline{\makebox[\remaining][s]{~Кузнецов Андрей Андреевич\hfill}}
  }$ \\[-0.5em]

  \textbf{Наименование темы ВКР}
  \uline{~Разработка веб-приложения для работы с программным пакетом высо\-коточного позиционирования RTKLIB\hfill} \\[-1em]

  \textbf{Наименование организации, где выполнена ВКР}
  \uline{~Университет ИТМО\hfill} \\[-2em]
}

\begin{center}
  \textbf{ХАРАКТЕРИСТИКА ВЫПУСКНОЙ КВАЛИФИКАЦИОННОЙ РАБОТЫ}
\end{center}

{
  \parindent0pt

  \textbf{1 Цель исследования}
  \uline{Создание веб-приложения для обеспечения взаимодействия пользователя с~программным пакетом RTKLIB, используемом во~встраиваемом решении, с~помощью кроссплат\-форменного графического интерфейса\hfill} \\[-1em]

  \textbf{2 Задачи, решаемые в ВКР} \\
  \uline{
    2.1 Изучить возможности программного комплекса RTKLIB;\hfill
  }\\
}

\clearpage
