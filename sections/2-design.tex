\mysection{ВЫБОР ПЛАТФОРМЫ ДЛЯ РАЗРАБОТКИ И~ПРОЕКТИРОВАНИЕ ВЕБ-ПРИЛОЖЕНИЯ}

\subsection{Платформа для разработки}

Разработка приложения будет осуществляться на платформе продуктов компании Emlid \cite{Emlid}: ГНСС модуля Reach \cite{Reach} и~ГНСС приёмника Reach~RS \cite{ReachRS}. Основой данных устройств являются вычислительный модуль Intel Edison и~плата, на которую установлен ГНСС модулю компании u-blox. Intel Edison работает под управлением GNU/Linux, что позволяет использовать многочисленные средства разработки, доступные для дистрибутивов данной операционной системы. \par

При разработке приложения на платформе указанных выше устройств важно учитывать следующий факт: несмотря на то, что Reach и~Reach~RS созданы на базе одного и~того же вычислительного модуля и~используют одинаковые приёмники u-blox, имеется ряд существенных различий в~аппаратном обеспечении данных устройств \cite{Reach, ReachRS}. Различия Reach и~Reach~RS, которые необходимо учесть при создании веб-приложения, указаны в~таблице \ref{tab:reach-vs-reachrs}.

\ctable[
  pos=h!,
  caption={~--~Различия Reach и~Reach~RS},
  label={tab:reach-vs-reachrs}
]{|l|*{2}{>{\centering\arraybackslash}m{2.3cm}|}}{}{
  \toprule
  \multicolumn{1}{|c|}{\textbf{Техническая/функциональная особенность}} & \textbf{Reach} & \textbf{Reach~RS} \\
  \midrule
  Встроенная батарея & нет & да \\
  \midrule
  Встроенная антенна & нет & да \\
  \midrule
  Встроенное радио & нет & да \\
  % \midrule
  % Физическая кнопка на корпусе & нет & да \\
  \midrule
  Возможность управления фотокамерой & да & нет \\
  \bottomrule
}

\subsection{Выбор инструментов разработки}

\subsection{Общая архитектура приложения}

\newpage