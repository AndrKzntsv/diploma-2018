\mysection{РАЗРАБОТКА ПРИЛОЖЕНИЯ}

\subsection{Подготовка окружения для разработки}

Как было сказано ранее, основой разрабатываемого приложения будет являться JavaScript-фреймворк Vue.js. Данный фреймворк позволяет начать работу над проектом максимально просто -- достаточно добавить на веб-страницу необходимую версию JavaScript-файла с~кодом Vue.js, используя тег \,\verb|<script>|.

Однако, для обеспечения удобства разработки и~возможности использования новейших возможностей языка JavaScript было решено использовать пакетный менеджер NPM [?] вместе с~компилятором Babel [?] и~утилитой Webpack [?] -- одним из наиболее популярных на данный момент [?] инструментом для сборки веб-приложений.


\subsubsection{Пакетный менеджер NPM}

NPM (аббр. \emph{Node Package Manager}) -- менеджер пакетов, входящий в~став программной платформы Node.js [?]. NPM существенно упрощает установку компонентов, необходимых для работы или сборки приложения.

При работе с~данным пакетным менеджером, особый интерес представляет файл \textbf{package.json}. Данный файл содержит массу информации о~разрабатываемом приложении: название, версия, описание, тип лицензии и~т.д. Но наиболее важным содержимым файла package.json являются зависимости -- список имён и~версий пакетов, требующихся для работы приложения.

% TODO Примеры содержимого package.json


\subsubsection{Babel}

Babel -- транспилер (англ. \emph{transpiler}), транслирующий код JavaScript стандартов ES2015 и~новее [?] в~код более ранних версий JavaScript.

Использование подобного инструмента позволит писать код, соответствующий последним стандартам JavaScript, не теряя при этом совместимость приложения со~старыми версиями веб-браузеров.

% TODO Пример работы Babel
% TODO Список поддерживаемых браузеров?


\subsubsection{Webpack}

Webpack -- система сборки для JavaScript-приложений, предназначенная, в первую очередь, для генерирования статических ресурсов на основе JavaScript-модулей и~их зависимостей.

Одним из основных преимуществ Webpack является его способность работать с~практически любыми типами ресурсов. Данная возможность обеспечивается дополнительно устанавливаемых \emph{загрузчиков} (англ. \emph{loaders}), которые, к~примеру, позволяют:
\begin{dashitemize}
  \item производить компиляцию JavaScript-файлов с~помощью Babel;
  \item осуществлять статический анализ кода с~помощью ESLint [?];
  \item минифицировать и обфусцировать код приложения;
  \item обрабатывать файлы с~расширением <<.vue>> (однофайловые компоненты Vue.js);
  \item производить трансляцию стилей, описанных на языке SCSS [?], в~CSS.
\end{dashitemize}

Стоит также отметить, что с~помощью Webpack можно существенно облегчить разработку ве-приложения. Благодаря специальным расширениям становится возможно запустить локальный HTTP-сервер, позволяющий просматривать и~отлаживать разрабатываемое приложение в~браузере компьютера, на котором ведётся разработка.



\subsection{Структура проекта. Артефакты сборки (?)}



\subsection{Глобальное хранилище данных. Однонаправленный поток данных}

Vuex -- библиотека-расширение для Vue.js приложений, позволяющая создавать глобальное хранилище данных, доступное для всех модулей и Vue-компонентов, входящих в состав проекта.

Идеи, лежащие в основе данной библиотеки, унаследованы от архитектуры Flux [?], представленная компанией Facebook. Vuex предоставляет шаблонный подход к управлению состояниями компонентов приложения, основанный на \emph{однонаправленном потоке данных}.


\subsubsection{Однонаправленный поток данных}

Взаимный обмен данными и событиями между моделями и представлениями может вызвать множество трудностей при увеличении числа компонентов. Асинхронные изменения и побочные эффекты могут существенно усложнить разработку и отладку, а также нарушить работу приложения.


\subsubsection{Модули хранилища данных}



\subsection{Модели представления}



\subsection{Модули общего назначения}



\subsection{Вспомогательные компоненты}

\newpage
