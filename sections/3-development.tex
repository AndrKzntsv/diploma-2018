\mysection{РАЗРАБОТКА ПРИЛОЖЕНИЯ}

\subsection{Подготовка окружения для разработки}

Как было сказано ранее, основой разрабатываемого приложения будет являться JavaScript-фреймворк Vue.js. Данный фреймворк позволяет начать работу над проектом максимально просто -- достаточно добавить на веб-страницу необходимую версию JavaScript-файла с~кодом Vue.js, используя тег \,\verb|<script>|.

Однако, для обеспечения удобства разработки и~возможности использования новейших возможностей языка JavaScript было решено использовать пакетный менеджер NPM [?] вместе с~компилятором Babel [?] и~утилитой Webpack [?] -- одним из наиболее популярных на данный момент [?] инструментом для сборки веб-приложений.


\subsubsection{Пакетный менеджер NPM}


\subsubsection{Babel}


\subsubsection{Webpack}



\subsection{Структура проекта. Артефакты сборки (?)}

\subsection{Глобальное хранилище данных. Модули глобального хранилища данных}

\subsection{Модели представления. Однонаправленный поток данных}

\subsection{Модули общего назначения}

\subsection{Вспомогательные компоненты}

\newpage
