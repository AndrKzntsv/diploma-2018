% !TeX spellcheck = ru_RU_yo
% !TEX program = xelatex

\documentclass[pta]{../../../scs-iam}

\usepackage{multirow}

\begin{document}

\newgeometry{
  top=20mm,
  right=15mm,
  bottom=20mm,
  left=20mm,
  bindingoffset=0cm
}

\thispagestyle{empty}

\begin{center}
  {
    \bfseries
    {
      \subnormal
      Министерство образования и науки Российской Федерации
    } \\[-0.5em]
    {
      \scriptsize
      ФЕДЕРАЛЬНОЕ ГОСУДАРСТВЕННОЕ АВТОНОМНОЕ ОБРАЗОВАТЕЛЬНОЕ УЧРЕЖДЕНИЕ ВЫСШЕГО ОБРАЗОВАНИЯ
    } \\[-0.25em]
    {
      \subnormal
      “САНКТ-ПЕТЕРБУРГСКИЙ НАЦИОНАЛЬНЫЙ ИССЛЕДОВАТЕЛЬСКИЙ \\[-0.5em]
      УНИВЕРСИТЕТ ИНФОРМАЦИОННЫХ ТЕХНОЛОГИЙ, \\[-0.75em]
      МЕХАНИКИ И ОПТИКИ”
    } \\[1em]
  }
\end{center}

\small

\begin{center}
  {
    \normalsize
    \textbf{О Т З Ы В}
  } \\[-0.25em]
  \textbf{РУКОВОДИТЕЛЯ}~$\underset{\text{\scriptsize (название практики)}}{\underline{\makebox[.42\textwidth][c]{производственной (преддипломной)}}}$~\textbf{ПРАКТИКИ}
\end{center}

{
  \parindent 0pt

  \textbf{Студент}
  $\underset{
    \text{\scriptsize (Фамилия, Имя, Отчество)}
  }{
    \underline{\makebox[.65\textwidth][l]{Кузнецов Андрей Андреевич}}
  }$
  \hfill
  \textbf{Группа №}
  \underline{\makebox[.1\textwidth][c]{\strut P4215~~}} \\[-1em]
  
  \textbf{Факультет}
  \uline{программной инженерии и компьютерной техники \hfill} \\[-1em]

  \textbf{Кафедра}
  \uline{информатики и прикладной математики \hfill} \\[-1em]

  \textbf{Направление подготовки (специальность)}
  \uline{09.04.01 \hfill} \\[-1em]

  \textbf{Место прохождения практики}
  \uline{ИП Николаев Денис Александрович \hfill} \\[-1em]
  
  \textbf{Должность практиканта}
  \uline{~программист \hfill} \\[-1em]
  
  \textbf{Тема индивидуального задания}
  \uline{<<Разработка архитектуры клиентской части веб-приложения для работы с устрой\-ствами Emlid Reach и Emlid Reach RS>> \hfill}

  \begin{center}
    \textbf{ОЦЕНКА ДОСТИГНУТЫХ РЕЗУЛЬТАТОВ}
  \end{center}

  \vskip -2em

  \ctable[
  pos=h!
  ]{|c|l|*{4}{>{\centering\arraybackslash}m{0.5cm}|}}{}{
    \toprule
    \multirow{2}{*}{\makecell[c]{\textbf{№}\\[-0.75em]\textbf{п/п}}} & \multirow{2}{*}{\makecell[c]{\textbf{Показатели}}} & \multicolumn{4}{c|}{\textbf{Оценка}} \\
    \cline{3-6}
    & & 5 & 4 & 3 & 2 \\
    \midrule
    1 & Объём и качество графических материалов & + &  &  & \\
    \midrule
    2 & \makecell[l]{Самостоятельное изучение документации\\[-0.5em]и спецификаций} & + &  &  & \\
    \midrule
    3 & Работа с системой управления проектом &  & + &  & \\
    \midrule
    4 & Использование информационных ресурсов & + &  &  & \\
    \midrule
    \multicolumn{2}{|c|}{\textbf{ИТОГОВАЯ ОЦЕНКА}} & \multicolumn{4}{c|}{отлично} \\
    \bottomrule
  }
}

\restoregeometry

\clearpage

\newgeometry{
  top=20mm,
  right=20mm,
  bottom=20mm,
  left=15mm,
  bindingoffset=0cm
}

\thispagestyle{empty}

{
  \parindent 0pt

  \textbf{Отмеченные достоинства:}
  \uline{Работа проведена со всей ответственностью и обязательностью. Ре\-зультаты работы представлены чётко, ясно и последовательно. \hfill} \\[-1em]

  \noindent\textbf{Отмеченные недостатки:}
  \uline{Отсутствуют. \hfill} \\[-1em]

  \textbf{Заключение:}
  \uline{Работа выполнена качественно. Уровень реализации соответствует заявленной идее. Работа заслуживает оценки <<отлично>>. \hfill} \\

  Руководитель практики \signature \\

  \datetemplate
}

\end{document}
