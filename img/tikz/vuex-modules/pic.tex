% !TEX program = xelatex
% !TeX spellcheck = ru_RU_yo

\documentclass[crop,tikz]{standalone}

\usepackage{fontspec}
\setmainfont{CMU Serif}

\usepackage{pgfplots}

\usepackage{polyglossia}
\setdefaultlanguage[spelling=modern]{russian}
\newfontfamily{\cyrillicfont}{PT Astra Sans}
\newfontfamily{\cyrillicfontsf}{PT Astra Sans}
\newfontfamily{\cyrillicfonttt}{PT Mono}

\usetikzlibrary{arrows.meta}

\tikzset{scheme part/.style={
  black,
  very thick,
  inner sep=0pt,
  draw=black,
  font=\huge,
  rounded corners=2mm
}}

\begin{document}

\begin{tikzpicture}
  \draw[opacity=0] (6.9,-6.1) grid (32.1,10.6);

  \draw[scheme part]
    (8,7) rectangle (19,9)
    node[align=center, pos=.5] {Информация об устройстве};

  \draw[scheme part]
    (20,7) rectangle (31,9)
    node[align=center, pos=.5] {Текущие настройки};

  \draw[scheme part]
    (8,4) rectangle (19,6)
    node[align=center, pos=.5] {Информация о RTK-системе};

  \draw[scheme part]
    (20,4) rectangle (31,6)
    node[align=center, pos=.5] {Информация о потоках данных};

  \draw[scheme part]
    (8,1) rectangle (19,3)
    node[align=center, pos=.5] {Беспроводные соединения};

  \draw[scheme part]
    (20,1) rectangle (31,3)
    node[align=center, pos=.5] {Изыскания};

  \draw[scheme part]
    (8,-5) rectangle (31,0)
    node[align=center, pos=.5, yshift=1.5cm] {Состояния компонентов интерфейса};

  \draw[scheme part]
    (9,-4) rectangle (19,-2)
    node[align=center, pos=.5] {Формы};

  \draw[scheme part]
    (20,-4) rectangle (30,-2)
    node[align=center, pos=.5] {Пагинация};

  \draw[
    ultra thick,
    dashed,
    rounded corners=2mm
  ] (7,-6) rectangle (32,10.5);
  
  \node[
    below,
    yshift=-0.4cm,
    font=\huge
  ] at (current bounding box.north) {Vuex};
\end{tikzpicture}
\end{document}