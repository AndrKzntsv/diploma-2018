% !TEX program = xelatex
% !TeX spellcheck = ru_RU_yo

\documentclass[crop,tikz]{standalone}

\usepackage{fontspec}
\setmainfont{CMU Serif}

\usepackage{polyglossia}
\setdefaultlanguage[spelling=modern]{russian}
\newfontfamily{\cyrillicfont}{PT Astra Sans}
\newfontfamily{\cyrillicfontsf}{PT Astra Sans}
\newfontfamily{\cyrillicfonttt}{PT Mono}

\usetikzlibrary{calc,arrows.meta}

\definecolor{darkgray}{RGB}{85,85,85}

\newcounter{treeline}

\newcommand{\treeroot}[1]{ % Title
  \node[above] at (0,1) {\texttt{#1}};

  \setcounter{treeline}{0}
}

\newcommand{\treeentry}[2]{ % Title, Level
  \draw[thick]
    (#2 - 1, -\value{treeline} * 2 + 1) --
    (#2 - 1, -\value{treeline} - \value{treeline} - 1) --
    (#2 + 1, -\value{treeline} - \value{treeline} - 1) node[right] {\texttt{#1}};

  \node at (#2 + 1, -\value{treeline} - \value{treeline} - 1) [yshift=-3pt] {\textbullet};

  \foreach \x in {1,...,#2}
  {
    \draw[thick]
      (\x - 1, -\value{treeline} * 2 + 1) --
      (\x - 1, -\value{treeline} - \value{treeline} - 1);
  }

  \stepcounter{treeline}
}

\newcommand{\treeentrycommented}[4]{ % Title, Level, Dummy node, Comment
  \draw[thick]
    (#2 - 1, -\value{treeline} * 2 + 1) --
    (#2 - 1, -\value{treeline} - \value{treeline} - 1) --
    (#2 + 1, -\value{treeline} - \value{treeline} - 1) node[right] (#3) {\texttt{#1}};

  \node at (#2 + 1, -\value{treeline} - \value{treeline} - 1) [yshift=-3pt] {\textbullet};
  
  \draw[
    gray,
    line width=3pt, line cap=round, dash pattern=on 0pt off 10pt
  ] (#3) -- (20, -\value{treeline} - \value{treeline} - 1) node[darkgray,right] {#4};

  \foreach \x in {1,...,#2}
  {
    \draw[thick]
      (\x - 1, -\value{treeline} * 2 + 1) --
      (\x - 1, -\value{treeline} - \value{treeline} - 1);
  }
  
  \stepcounter{treeline}
}

\begin{document}

\begin{tikzpicture}
  \fontsize{30}{36}\selectfont
  
  \treeroot{\textbf{static-src/}}
    \treeentrycommented{\textbf{build/}}{1}{webpack}{Конфигурационные файлы Webpack}
      \treeentry{*.conf.js}{2}
    \treeentrycommented{\textbf{src/}}{1}{src}{Исходные файлы JavaScript-приложения}
      \treeentrycommented{\textbf{assets/}}{2}{assets}{Ресурсы приложения: стили и изображения}
        \treeentry{\textbf{images/}}{3}
          \treeentry{*.png, *.svg}{4}
        \treeentry{\textbf{styles/}}{3}
          \treeentry{*.css, *.scss}{4}
      \treeentry{\textbf{components/}}{2}
        \treeentrycommented{\textbf{common/}}{3}{common}{Вспомогательные компоненты}
          \treeentry{Alert.vue}{4}
          \treeentry{Button.vue}{4}
          \treeentry{...}{4}
        \treeentrycommented{\textbf{tabs/}}{3}{tabs}{Модели представления и представления вкладок}
          \treeentry{\textbf{CameraControl/}}{4}
            \treeentry{index.vue}{5}
            \treeentry{style.scss}{5}
            \treeentry{template.html}{5}
          \treeentry{...}{4}
        \treeentry{BatteryIcon.vue}{3}
        \treeentry{Logo.vue}{3}
        \treeentry{...}{3}
      \treeentrycommented{\textbf{modules/}}{2}{modules}{Модули приложения}
        \treeentry{*.js}{3}
      \treeentrycommented{\textbf{store/}}{2}{vuex}{Хранилище Vuex}
        \treeentry{\textbf{modules/}}{3}
        \treeentry{index.js}{3}
      \treeentrycommented{main.js}{2}{main}{\textbf{Точка входа}}
    \treeentrycommented{\textbf{test/}}{1}{tests}{Модульные тесты}
      \treeentry{\textbf{specs/}}{2}
      \treeentry{index.js}{2}
      \treeentry{karma.conf.js}{2}
    \treeentry{index.html}{1}
\end{tikzpicture}

\end{document}